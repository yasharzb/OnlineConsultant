\section{اهداف پروژه}

\subsection{مقدمه}
با پیشرفت تکنولوژی و افزایش ضریب نفوذ رایانه و اینترنت در زندگی روزمره، پتانسیل ایجاد تغییر در برخی از کارهای روزمره انسان به وجود آمده است. اعضای جامعه ما برای انجام بسیاری از کارها (مانند پرداخت قبض، شرکت در کلاس‌های آموزشی و ...) در گذشته می‌بایست به مکان‌های خاصی مراجعه می‌کردند و بعضاً بخش عظیمی از زمان خود را در صف و در انتظار می‌گذراند. اما با افزایش امکانات و خدمات اینترنت، بسیاری از این اتلاف وقت‌ها کاهش یافته و امروزه می‌توان با استفاده از اینترنت و تنها با چند کلیک، بسیاری از کارهای زمان‌بر را به‌راحتی انجام داد. \\
البته لازم به ذکر است که این تغییر در برخی از حوزه‌ها، مانند مراجعه به پزشک و مشاور، به‌طور کامل جا نیافتاده است و اکثر افراد امروزه تمایل به مراجعه حضوری برای معاینات پزشکی و مشاوره را دارند. در سالیان اخیر و با معرفی چندین سرویس در این حوزه (مانند اسنپ و زوپ) و همچنین بیماری کرونا و ایجاد محدودیت‌های مربوط به پروتکل‌ها، این موضوع نیز در حال تغییر می‌باشد اما هنوز هم تعداد و کیفیت این سرویس‌ها جای پیشرفت دارد. \\
برای همین در این پروژه قصد داریم تا یک سامانه اختصاصی برای مشاوره روانشناسی و پزشکی طراحی و پیاده‌سازی کنیم که در آن مشاوران، روانشناسان، روان‌پزشکان، و پزشکان حاذق به ارائه خدمات به متقاضیان مشاوره بپردازند. 

\subsection{اهداف کلی}
یکی از اهداف اصلی این سامانه، راحت‌تر کردن ایجاد ارتباط میان کاربر و مشاور مناسب است. به‌طور سنتی، هر فرد با توجه به محل زندگی، انتخاب محدودی بین مشاوران مدنظر داشت و برای گرفتن وقت مشاوره می‌بایست از همان تعداد محدود فردی را به‌عنوان مشاور انتخاب می‌کرد. در شهرهای کوچک و کم برخوردار نیز معمولاً مشاور و پزشکان مطرح زیادی وجود ندارند و این موضوع باعث به وجود آمدن نوعی بی‌عدالتی در حق ساکنان این نوع شهرها می‌شود. با استفاده از این سامانه، این مشکل به‌طور کامل حل می‌شود. کاربران می‌توانند با استفاده از این سامانه و با فیلتر کردن انتخاب‌های خود، بهترین مشاور را انتخاب کنند. \\
هدف دیگر سامانه کاهش اتلاف زمان، انرژی و هزینه جهت مراجعه حضوری برای مشاوره می‌باشد. با استفاده از این سامانه، فرآیند مشاوره به‌طور کامل به‌صورت مجازی (با استفاده از چت باکس و یا به‌صورت ویدیویی) انجام می‌شود و کاربر لازم نیست که به‌صورت حضوری به مطب مشاور مراجعه کند. این کار، علاوه بر کاهش هزینه‌های رفت آمد، منجر به کاهش ترافیک و مصرف سوخت نیز می‌گردد. همچنین کاربر لازم نیست مدت‌زمان طولانی‌ای را برای رسیدن به مطب مشاوره صرف نماید. این سامانه به مشاوران نیز کمک می‌کند تا بتوانند از زمان اضافه خود استفاده بهینه داشته باشند. \\
هدف دیگر این سامانه ایجاد نوعی رقابت سالم میان مشاوران برای ارتقای سطح خدمات مشاوره می‌باشد. برای این منظور یک سیستم امتیازدهی دقیق برای هر جلسه مشاوره تدارک دیده خواهد شد تا یک معیار سنجش برای هر مشاور ایجاد شود و این منجر به ایجاد نوعی رقابت بین مشاوران برای ارتقای سطح عملکرد خود خواهد شد. 

\subsection{خلاصه اهداف}
\begin{itemize}
\item تحلیل، طراحی و پیاده‌سازی سامانه مشاوره آنلاین
\item ایجاد امکان دریافت مشاوره برای همه ساکنین ایران
\item ایجاد امکان عضویت مشاوران سرتاسر کشور برای شناساندن بیشتر خود
\item کاهش اتلاف وقت و هزینه صرف‌شده جهت مراجعه حضوری برای مشاوره
\item ایجاد فضای رقابت سالم میان مشاوران جهت بهبود عملکرد با استفاده از سیستم امتیازدهی
\end{itemize}