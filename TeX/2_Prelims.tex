\section{پیش‌زمینه پروژه}
\subsection{عوامل طرح مسئله}
سه عامل اصلی شروع یک پروژه تحلیل، طراحی و پیاده‌سازی سیستم اطلاعاتی موارد زیر می‌توانند ‌باشند:
\begin{itemize}
\item مشکل یا مسئله (Problem)ای که یک نیاز را بوجود می‌آورد.
\item فرصت (Opportunity) برای بهبود عملکرد
\item دستور از مراجع بالا (Directive) برای ایجاد تغییرات
\end{itemize}
دو عامل اصلی این پروژه، از جنس مشکل و فرصت می‌باشد. مشکل و مسئله این است که با توجه به حضور متمرکز اکثر مشاوران و متخصصان مطرح در شهرهای بزرگ، دسترسی نداشتن آحاد مردم به مشاوران مناسب، اتلاف وقت و هزینه جهت مراجعه حضوری، و همچنین در دسترس نبودن مشاور در هر زمان وجود چنین سامانه‌ای در سطح کشور احساس می‌شود. از طرفی با توجه به گسترش بیماری کرونا و نبود بستر مناسب جهت مشاوره آنلاین، این فرصت به وجود آمده تا یک سامانه مشاوره آنلاین بتواند عملکرد مناسبی داشته باشد. در این بخش به بررسی بیشتر این مشکل و فرصت می‌پردازیم.


\subsection{مشکل}
همان‌گونه که در بخش‌های قبلی اشاره شد، مشکل دسترسی نداشتن ساکنان شهرهای کم برخوردار به مشاوران مجرب و همچنین اتلاف وقت و هزینه برای حضور در مطب مشاوره، یکی از عوامل ایجاد سامانه مشاوره آنلاین است. با راه‌اندازی این سامانه، افراد خواهان مشاوره صرفاً با یک ثبت‌نام و یک جستجوی ساده، مشاور موردنظر خود را پیدا می‌کنند و حتی در صورت نیافتن مشاور مدنظر، با استفاده از سیستم جستجوی پیشرفته و یا سیستم توصیه‌گر مشاور مناسب را می‌یابند. همچنین با توجه به آنلاین بودن سامانه و قرار مشاوره، افراد می‌توانند در اکثر ساعات روز، از خدمات مشاوره بهره‌مند شوند. همچنین ساکنان شهرهای کوچک و کم برخوردار نیز از امکان مشاوره با متخصصان مجرب برخوردار می‌گردند. این سامانه می‌تواند از اتلاف وقت و هزینه برای رفت‌وآمد به مطب مشاوره را بکاهد و تأثیری (هرچند اندک) بر روی کاهش استفاده از وسایل نقلیه و ترافیک بگذارد.

\subsection{فرصت}
با شروع همه‌گیری بیماری کرونا، یک فرصت برای کسب‌وکارها در اکثر زمینه‌ها (آموزش، خدمات و ...) فراهم گردید تا با ایجاد بستر آنلاین برای ارائه خدمات، کسب‌وکار خود را گسترش دهند. برای مثال دانشگاه شریف قصد ایجاد یک بستر آنلاین برای ارائه درس‌ها و آموزش‌های خود داشت اما این پروژه قبل از دوران همه‌گیری، به‌کندی پیش می‌رفت. با شروع همه‌گیری، این پروژه با سرعت بیشتری پیگیری شد و اکنون اکثر دروس دانشگاه در این بستر ارائه می‌شود. \\
با توجه به فرصت پیش‌آمده، می‌توان با ارائه یک سامانه و سرویس اختصاصی مشاوره آنلاین، هم به نیاز به وجود آمده پاسخ مناسب داد و ازنظر مالی و اقتصادی، عملکرد مناسبی داشت.

