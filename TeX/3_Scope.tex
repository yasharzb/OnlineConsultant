\section{گستره‌ی پروژه}
\subsection{ذینفعان سامانه}
\subsubsection{مالکین}
مالک این پروژه جناب آقای دکتر حیدرنوری، استاد درس تحلیل و طراحی سیستم‌ها، می‌باشد. 

\subsubsection{کاربران}
سه دسته کاربران سامانه را تشکیل می‌دهند:
\begin{itemize}
\item متقاضیان مشاوره: افرادی که می‌خواهند از خدمات مشاوره آنلاین (چه به‌صورت متنی و چه به‌صورت ویدیویی) استفاده نمایند. آن‌ها می‌توانند با استفاده از سیستم جستجو (ساده و پیشرفته) و لیست مشاوران، مشاور مناسب خود را پیدا کرده و درخواست وقت مشاوره بنمایند.
\item مشاوران: این افراد متخصصان مشاوره در زمینه روانشناسی و پزشکی و ... هستند. این کاربران زمان حضور خود جهت مشاوره را مشخص می‌نمایند و در صورت تأیید، به فرد متقاضی به‌صورت متنی یا ویدیویی، مشاوره می‌دهند. 
\item مدیران سامانه: این افراد به مدیریت سامانه و بررسی صلاحیت مشاوران بر اساس مدارک ارسالی آن‌ها می‌پردازند. همچنین این افراد بر کاربران نظارت دارند.
\end{itemize}
\subsubsection{تحلیلگران و طراحان سامانه}

\subsubsection{مدیر پروژه}

\subsubsection{فراهم‌کنندگان سرویس خارجی}
در این پروژه از پلتفرم \href{https://meet.jit.si/}{jitsi} برای تهیه بستر ملاقات ویدیوی استفاده خواهد شد.

\subsection{داده‌ها}
داده‌های زیر در این سامانه ذخیره می‌گردند:

\subsubsection{اطلاعات کاربران}
		همه کاربران در این سامانه دارای پروفایل هستند. در پروفایل کاربران عادی یا متقاضیان مشاوره، اطلاعاتی مانند نام، نام خانوادگی، جنسیت، شماره تلفن، سال تولد ذخیره می‌شود. در پروفایل مشاوران، اطلاعاتی مانند نام، نام خانوادگی، جنسیت، حوزه‌های تخصص، اطلاعات مربوط به مدرک، شماره عضویت در نظام پزشکی، شماره تماس، آدرس مطب (در صورت وجود) ذخیره می‌گردد.
همچنین امتیازاتی که هر کاربر به هر مشاور (پس از مشاوره) داده است نیز ذخیره می‌گردند. البته این اطلاعات به‌گونه‌ای ذخیره خواهند شد که حریم شخصی هر کاربر حفظ شود.

\subsubsection{اطلاعات جلسه مشاوره}
اطلاعات مربوط به هر جلسه مشاوره مانند مدت زمان، تاریخ، نام مشاور و کاربر عادی، موضوع مشاوره و تراکنش بانکی انجام شده مربوط به این جلسه ذخیره می‌گردند.

\subsubsection{تراکنش‌ها}
همه تراکنش‌های مربوط به سامانه مانند جستجوها، بازدید از پروفایل مشاوران، همه پرسش‌های کاربران و سایر تراکنش‌ها ذخیره می‌گردند.

\subsection{امکانات}
\subsubsection{امکانات مربوط به متقاضیان مشاوره}
\begin{itemize}
    \item امکان ایجاد حساب کاربری با ثبت اطلاعات مورد نیاز (نام، نام خانوادگی، شماره‌ تلفن، رمز عبور و ...)
	\item امکان تعریف مشکلات و سابقه بیماری
	\item امکان جستجوی مشاور بر اساس حوزه کاری و همچنین امکان استفاده از سیستم جستجوی پیشرفته
	\item امکان مشاوره با مشاوران سراسر ایران (در صورت حضور مشاور در سامانه) به صورت متنی (از طریق پیامرسان داخل سامانه) و ویدیویی (از طریق پلتفرم‌های ارتباط ویدیویی مانند jitsi) بهمراه ارسال توضیحات قبل از مشاوره
	\item امکان مشاهده زمان‌های در دسترس بودن هر مشاور به‌همراه هزینه ویزیت
	\item امکان امتیازدهی و نظردهی به هر مشاور پس از مشاوره
	\item امکان کنسل کردن زمان مشاوره 24 ساعت قبل از زمان جلسه مشاوره و مسترد شدن بخشی از مبلغ 
	\item امکان مشاهده لیست مشاوران بر اساس تخصص و امتیاز کسب‌شده
	\item امکان تهیه لیست مشاوران مورد علاقه 
    \item امکان مشاهده نوبت‌های کاربر، اطلاعات مربوط به مشاوره‌های قبلی، میزان شارژ حساب کاربری، تراکنش‌های انجام‌شده
\end{itemize}

\subsubsection{امکانات مربوط به مشاوران}
\begin{itemize}
    \item امکان ثبت‌نام با ثبت مدارک موردنیاز (نام و نام خانوادگی، کد ملی، مجوز طبابت، مدرک تحصیلی، شماره تماس و ...) و تایید مدیران سایت
	\item امکان مشاوره‌دادن به کاربران متقاضی مشاوره
	\item امکان تعیین بهای ویزیت و زمان‌های دردسترس برای مشاوره
	\item امکان مشاهده لیست جلسات مشاوره آتی
	\item امکان کنسل کردن زمان مشاوره 24 ساعت قبل از زمان مشاوره (و مسترد شدن هزینه به کاربر)
	\item امکان مشاهده سوابق مشاوره و مشکلات فردی هر فرد متقاضی مشاوره
	\item امکان گزارش کاربران متخلف به مدیران
	\item امکان مشاهده اطلاعات مربوط مشاوره‌های قبلی، میزان شارژ حساب کاربری، تراکنش‌های انجام شده
\end{itemize}

\subsubsection{امکانات مربوط به مدیران سامانه}
\begin{itemize}
\item امکان تایید ثبت‌نام مشاوران و پزشکان با توجه به مدارک ارسال‌شده
\item امکان حذف کاربران متخلف
\end{itemize}
\subsubsection{امکانات مازاد (backlog)}
در صورت داشتن زمان یا بودجه اضافی، می توان این موارد را نیز در سیستم پیاده‌سازی کرد:
\begin{itemize}
\item سیستم توصیه‌گر مشاور (بر اساس مشاوره‌ها، امتیاز ثبت‌شده توسط کاربر برای هر مشاوره، و همچنین مشکلات ثبت شده بیمار)
\item پیاده‌سازی بخش پرسش و پاسخ رایگان: در این بخش بیماران می‌توانند سوالات کلی خود را بپرسند و پزشکان و مشاوران به این سوالات پاسخ دهند. هر کدام از بیماران محدودیتی در تعداد ارسال سوالات دارند و همچنین برای هر پزشک پاسخ‌دهنده، امتیاز مثبت ثبت می‌گردد.
\end{itemize}

\subsection{گستره مکانی}
همه مشاوران متخصص در سرتاسر ایران توانایی عضویت در این سامانه و مشاوره‌دادن به کاربران را دارند. همچنین همه افراد نیز می‌‌توانند از خدمات مشاوره این سامانه استفاده نمایند.