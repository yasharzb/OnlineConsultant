\section{رویکرد پروژه}
رویکرد پروژه به صورت محصول‌محور خواهد بود و از روش توسعه چابک نرم‌افزار استفاده خواهیم کرد. 
این رویکرد مبتنی بر تکرار است و امکانات محصول به تدریج به آن اضافه خواهند شد و پیشرفت محصول قابل لمس و مشاهده خواهد بود. 
این روش برنامه‌ریزی تطبیقی، توسعه، تحویل تکاملی و رویکرد زمان بسته‌بندی تکرارشونده را ارتقا بخشیده و پاسخ‌های سریع و انعطاف‌پذیر برای انجام تغییرات در محصول را تقویت می‌کند.

\subsection{مسیر پروژه}

در فاز صفر، مسئله و گستره‌ی آن را به طور دقیق مشخص می‌کنیم.

در فاز یکم به به تحلیل مسئله
\footnote{\lr{Problem Analysis}}
و بررسی نیازمندی‌ها
\footnote{\lr{Requirements Analysis}}
می‌پردازیم.
سپس با توجه به نیازمندی‌ها و از روی اصول مشخص، نمودار مورد کاربرد را بدست خواهیم آورد. در این فاز باید مکانیزم‌های سامانه به گونه‌ای تعریف شوند که حداکثر رضایت ذی‌النفعان مختلف جلب شود.

در فاز دوم به مدل‌سازی فرآیندها و بانک‌های اطلاعاتی یا به عبارت دیگر، طراحی منطقی
\footnote{\lr{Logical Design}}
می‌پردازیم. در این فاز با استفاده از نیازمندی‌های جمع‌آوری شده به مشخص کردن موجودیت‌ها، صفات و نوع ارتباط میان آن‌ها خواهیم پرداخت. همچنین محدودیت‌های پیاده‌سازی در این فاز مشخص خواهند شد.

در فاز سوم با استفاده از نمودار‌های تولید شده در فاز‌های قبلی به پیاده‌سازی و ارزیابی سامانه اطلاعاتی با بهره گرفتن از شیوه‌ی چابک اسکرام می‌پردازیم.

\newpage
\subsection{تحویل‌دادنی‌ها}
تحویل‌دادنی‌ها (خروجی) هر فاز از پروژه به صورت زیر خواهد بود:

\begin{itemize}
    \item فاز صفر
    
    پیشنهادنامه‌ی پروژه
    \item فاز یکم
    
    نمودار مورد کاربرد
    \footnote{\lr{Use Case}}
    
    سناریوهای سامانه
    \item فاز دوم
    
    نمودار جریان‌داده‌ها
    \footnote{\lr{Data flow Diagram}} 
    (مدل‌سازی فرآیند‌ها)
    
    نمودار داده رابطه‌ای (مدل‌سازی بانک‌های اطلاعاتی)
    \item فاز سوم
    
    نسخه‌نهایی محصول
    
    مستندات پروژه
    
\end{itemize}
